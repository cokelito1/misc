\documentclass[../main.tex]{subfiles}

\begin{document}

\subsection{Topologia}
\begin{problem}
  Sea $(X, \topo)$ un espacio topologico tal que $B \subset X$ sea un subconjunto denso en $X$. Si
  $A$ es un conjunto denso en $(B, \topo_{B})$, donde $\topo_{B}$ es la topologia inducida de $X$ en $B$, demostrar que $A$ es denso en $(X, \topo)$
\end{problem}
\begin{solution}
  Sea $\theta \in \topo, \theta \neq \emptyset$, dado que $B$ es denso en $X$ tenemos que
  \begin{equation}
    \label{eqn_fst1}
    \theta \cap B \neq \emptyset
  \end{equation}
  Pero sabemos que $\theta \cap B \in \topo_{B}$ y de (\ref{eqn_fst1}) sabemos que es no vacio, por lo tanto dado que $A$ es denso en $(B, \topo_{B})$ tenemos que
  \begin{equation*}
    (\theta \cap B) \cap A \neq \emptyset
  \end{equation*}

  Pero $\theta \cap B \cap A \subset \theta \cap A$, por lo tanto $\theta \cap A \neq \emptyset$.
  Lo que significa que $A$ es denso en $(X, \topo)$
\end{solution}

\end{document}
