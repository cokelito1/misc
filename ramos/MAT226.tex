\documentclass[../main.tex]{subfiles}

\begin{document}

\subsection{Teoria de la medida}
\begin{problem}
  Sea $\R$ con la algebra de Borel, si $f : \R \to \R$ es una funcion continua, entonces $f$ es medible.
\end{problem}
\begin{solution}
  Sea $\alpha \in \R$, luego $(\alpha, \infty)$ es un conjunto abierto, dado que $f^{-1}((\alpha, \infty))$ es abierto, pues $f$ es continua, tenemos que $f^{-1}((\alpha, \infty)) \in B$, para todo $\alpha$, por lo tanto $f$ es medible.
\end{solution}

\begin{solution}
Sea $x \in X \setminus (\bigcup_{A \in \mathcal{A}} A)$
\begin{equation*}
  x \in X \land x \notin \bigcup_{A \in \mathcal{A}} A \iff \forall A \in \mathcal{A}, x \notin A \iff \forall A \in \mathcal{A}, x \in X \setminus A \iff x \in \bigcap_{A \in \mathcal{A}} X \setminus A
\end{equation*}

Por lo tanto tenemos
\begin{equation*}
  X \setminus (\bigcup_{A \in \mathcal{A}} A) = \bigcap_{A \in \mathcal{A}} X \setminus A
\end{equation*}
\end{solution}

\begin{problem}
  Sea $|\cdot|$ la medida exterior sobre $\R$. Demuestre que si $a, b \in \R$ con $a < b$, entonces
  \begin{equation*}
    |(a, b)| = |[a, b)| = |(a, b]| = b - a
  \end{equation*}
\end{problem}
\begin{solution}
  Notemos que $[a, b) \subset [a, b]$ por lo que tenemos que
  \begin{equation*}
    |[a, b)| \leq |[a, b]| = b - a
  \end{equation*}

  Notemos tambien que $[a, b] = [a, b) \cup \{b\}$ por lo que por la $\sigma$-subaditividad tenemos que
  \begin{equation*}
    b - a = |[a, b]| = |[a, b) \cup \{b\}| \leq |[a, b)| + |\{b\}| = |[a, b)|
  \end{equation*}

  y por lo tanto tenemos que
  \begin{equation*}
    |[a, b)| = b - a
  \end{equation*}

  Analogamente para el resto con las siguientes igualdades
  \begin{align*}
    [a, b] &= (a, b) \cup \{a, b\}\\
    [a, b] &= (a, b] \cup \{a\}
  \end{align*}
\end{solution}

\begin{problem}
Sean $A, B \subset \R$ y $|B| = 0$. Demostrar que $|A \cup B| = |A|$
\end{problem}
\begin{solution}
  Notemos que $A \subset A \cup B$, entonces tenemos que
  \begin{equation*}
    |A| \leq |A \cup B|
  \end{equation*}

  Tambien tenemos, por la subaditividad de la medida exterior que
  \begin{equation*}
    |A \cup B| \leq |A| + |B| = |A|
  \end{equation*}

  Por lo tanto tenmeos que
  \begin{equation*}
    |A| = |A \cup B|
  \end{equation*}
\end{solution}

\begin{problem}
Demostrar que $|tA|$ = $|t||A|$, para $t \in \R$
\end{problem}
\begin{solution}
  Si $t = 0$ entonces el resultado es trivial pues $0 \cdot A = \{0\}$ lo cual es contable por lo tanto $|0 \cdot A| = 0$.

  Si $t \neq 0$ entonces, dada un cubrimiento abierto $\{I_{n}\}_{n \in \N}$ de $A$ tenemos que
  \begin{equation*}
    tA \subset \bigcup_{n \in \N} tI_{n}
  \end{equation*}

  Por lo que
  \begin{equation*}
    |tA| \leq \sum_{n \in \N} \ell(tI_{n}) = \sum_{n \in \N} |t|\ell(I_{n}) =|t| \sum_{n \in \N}\ell(I_{n})
  \end{equation*}
  Tomando infimo sobre los recubrimientos de $A$ obtenemos
  \begin{equation*}
    |tA| \leq |t||A|
  \end{equation*}
\end{solution}

\end{document}
