\documentclass[../main.tex]{subfiles}

\begin{document}

\subsection{Ecuaciones no lineales}
\begin{problem}
  Considere la siguiente funcion
  \begin{align*}
    &f : [0, 1] \to \R\\
    &x \mapsto x^{2} - 3x + 2 - e^{x}
  \end{align*}
  Demuestre que $f$ tiene una sola ra\'iz y aproxime su valor usando tres iteraciones del metodo de bisecci\'on.
\end{problem}

\begin{solution}
  Notar que por algebra de funciones continuas la funcion $f$ es continua, es m\'as esta, es $\mathcal{C}^{\infty}$.

  Notemos que $f(0) = 1$ y $f(1) = -e < 0$, por lo tanto dado que $f(0)\cdot f(1) = -e < 0$ por el teorema de Bolzano tenemos que existe al menos una raiz en $[0, 1]$.

  Calculemos la derivada de $f$
  \begin{equation*}
    f'(x) = 2x - 3 - e^{x}
  \end{equation*}

  Notar que para $x \in [0,1]$ se tiene que
  \begin{equation*}
    2x - 3 - e^{x} \leq 2 - 3 - e = -1 - e < 0
  \end{equation*}

  y por tanto la funcion es estrictamente decreciente, por lo tantola raiz es unica.
\end{solution}

\end{document}
