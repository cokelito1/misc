\documentclass[../main.tex]{subfiles}
\begin{document}

\begin{solution}
Asumiremos que la barra parte a una distancia $d$ de la resistencia.

Dado que la velocidad de la barra es constante tenemos que la posicision en funcion del tiempo
es
\begin{equation*}
  x(t) = d + v_{0}t
\end{equation*}

Dado que el flujo es ortogonal a la superficie considerada tenemos lo siguiente
\begin{equation*}
\phi(t) = B_{0}Lx(t)
\end{equation*}

entonces tenemos que
\begin{equation*}
  \phi'(t) = B_{0}Lx'(t) = B_{0}v_{0}L
\end{equation*}

Por la ley de faraday tenemos que se genera corriente, la cual genera un campo magnetico que se opone al cambio en el flujo es decir
\begin{equation*}
  \fem = -\phi'(t)
\end{equation*}
Entonces la direccion del campo magnetico que genera la FEM es $-\hat{k}$.
De lo que se deduce que la direccion de la FEM es $-\hat{j}$.

Por ley de ohm sabemos que
\begin{equation*}
\frac{-B_{0}v_{0}L}{R} = \frac{\fem}{R} = I
\end{equation*}
\end{solution}
\begin{solution}
  El campo magnetico generado por el cable infinito a su derecha es
  \begin{equation*}
    B(r, t) = \frac{\mu_{0}I(t)}{2\pi r} (-\hat{k})
  \end{equation*}

  Entonces tenemos que el flujo es
  \begin{equation*}
    -\int_{0}^{a}\int_{a}^{2a} \frac{\mu_{0}I}{2\pi r} dr dy = -\frac{\mu_{0} I a}{2\pi} \ln(2)
  \end{equation*}
  No existe FEM inducida pues no depende del tiempo.

  Ahora si suponemos que $I(t) = I_{0}\sin(\omega t)$
  \begin{equation*}
    \phi(t) = -\int_{0}^{a}\int_{a}^{2a} \frac{\mu_{0}I_{0} \sin(\omega t)}{2\pi r} dr dy = -\frac{\mu_{0} I_{0} \sin(\omega t) a}{2\pi} \ln(2)
  \end{equation*}
  Por lo que tenemos que
  \begin{equation*}
    \phi'(t) = -\frac{\mu_{0} I_{0} \omega a \cos(\omega t)}{2\pi} \ln(2)
  \end{equation*}

  Entonces la FEM inducida es
  \begin{equation*}
    \fem = \frac{\mu_{0} I_{0} \omega a \cos(\omega t)}{2\pi} \ln(2)
  \end{equation*}
\end{solution}
\begin{solution}
  El campo generado por el cable por debajo de este en el tiempo $t$ viene dado por
  \begin{equation*}
    B(r, t) = \frac{\mu_{0}I(t)}{2\pi r} = \frac{\mu_{0} (a + bt)}{2\pi r} (-\hat{k})
  \end{equation*}

  Luego el flujo viene dado por
  \begin{equation*}
    \phi(t) = \int_{-h}^{-h - \omega}\int_{0}^{L} \frac{\mu_{0} (a + bt)}{2\pi r} dx dr = \frac{\mu_{0} (a + bt)}{2\pi} L \ln(\frac{-h - \omega}{-h})
  \end{equation*}

  Luego tenemos que
  \begin{equation*}
    \fem = -\phi'(t) = - \frac{\mu_{0} b}{2\pi} L \ln(\frac{-h - \omega}{-h})
  \end{equation*}
\end{solution}
\begin{solution}
  Notemos que el flujo a traves de la ``espira'' viene dado por
  \begin{equation*}
    \phi(t) = 2.5 \cdot \ell \cdot x(t)
  \end{equation*}

  Entonces
  \begin{equation*}
    \fem = -2.5 \cdot 1.2 \cdot 2 = -6 [V]
  \end{equation*}

  Por lo tanto
  \begin{equation*}
    I_{ind} = \frac{\fem}{R} = 1 [A]
  \end{equation*}

  Luego tenemos que
  \begin{equation*}
    F_{mag} = \int_{0}^{l} 1 dl \times B = 2.5 \cdot \ell \cdot \hat{j} = -3 \hat{i} [N]
  \end{equation*}

  Luego la fuerza requerida es $3 \hat{i} [N]$
  \begin{equation*}
    P = \fem \cdot I = 6 [W]
  \end{equation*}
\end{solution}

\end{document}
