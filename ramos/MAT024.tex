\documentclass[../main.tex]{subfiles}

\begin{document}

\subsection{Teorema de Stokes}
\begin{problem}
Usando el teorema de Stokes, calcular la integral de linea $\oint_{C} x^{2}y^{3} dx + dy + z dz$ donde $C$ es la curva $x^{2} + y^{2} = R^{2}, z = 0$ con $R > 0$, recorrida en sentido antihorario
\end{problem}
\begin{solution}
  Notemos que la curva $C$ es cerrada, simple y suave. Esta curva encierra a la superficie $S: x^{2} + y^{2} \leq R^{2}, z = 0$ entonces por el teorema de Stokes tenemos que
  \begin{equation*}
    \iint_{S} \nabla \times F dS = \oint_{C} F dr
  \end{equation*}
  Calculemos $\nabla \times F$
  \begin{equation*}
    \nabla \times F = \curl{x^{2}y^{3}}{1}{z}
  \end{equation*}

  Consideremos la siguiente $\mathcal{C}^{\infty}$ parametrizacion de $S$
  \begin{gather*}
    \phi : [0, 2\pi] \times [0, R] \to \R^{3}\\
    (\theta, r) \mapsto (r \cos \theta, r \sin \theta, 0)
  \end{gather*}

  Donde tenemos que
  \begin{equation*}
    \hat{n} = \phi_{\theta} \times \phi_{r}
  \end{equation*}

  \begin{gather*}
    \phi_{\theta} = (-r \sin \theta, r \cos \theta, 0)\\
    \phi_{r} = (\cos \theta, \sin \theta, 0)
  \end{gather*}

  Por lo tanto el vector normal es
  \begin{equation*}
    \hat{n} = \crossprod{-r \sin \theta}{r \cos \theta}{0}{\cos \theta}{\sin \theta}{0} = (0, 0, -r)
  \end{equation*}

  Por lo tanto tenemos que
  \begin{equation*}
    \oint_{C} F \cdot dr = \iint_{D} (0, 0, 3r^{2}\cos^{2}(\theta) r^{2} \sin^{2}(\theta)) \cdot (0, 0, -r) dA
  \end{equation*}

  Calculemos la integral, donde el dominio es el dominio de la parametrizacion entonces
  \begin{align*}
    \int_{0}^{2\pi}\int_{0}^{R} -3r^{5}\cos^{2}(\theta)\sin^{2}(\theta) drd\theta &= -\frac{R^{6}}{2} \int_{0}^{2\pi}\cos^{2}(\theta)\sin^{2}(\theta) d\theta\\
                                                                                  &= -\frac{R^{6}}{2} \int_{0}^{2\pi} \cos^{2}(\theta)(1 - \cos^{2} \theta) d\theta\\
                                                                                  &= -\frac{R^{6}}{2} (\pi - \int_{0}^{2\pi} \cos^{4} \theta d \theta)\\
    &= -\frac{R^{6}\pi}{8}
  \end{align*}
\end{solution}

\begin{problem}
Calcule $\oint_{C} x \sin x - 2y^{2} dx + y \cos y - 2z dy + \tan z - 2x dz$ donde $C$ es la interseccion de $4x^{2} + 5y^{2} + z^{2} = 36$ con $z = 2y$
\end{problem}
\begin{solution}
  Notemos que $C$ es una curva cerrada, simple y suave. Podemos ocupar el teorema de Stokes por lo
  tanto
  \begin{equation*}
    \oint_{C} F dr = \iint_{S} (\nabla \times F) dS
  \end{equation*}

  Calculemos el rotor
  \begin{equation*}
    \nabla \times F = \curl{x \sin x - 2y^{2}}{y \cos y - 2z}{\tan z - 2x} = (2, 2, 4y)
  \end{equation*}

  Consideremos la siguiente parametrizacion, con las variaciones por determinar
  \begin{equation*}
    \phi(x, y) = (x, y, 2y)
  \end{equation*}

  donde sabemos que la normal es
  \begin{equation*}
    \hat{n} = (-f_{x}, -f_{y}, 1) = (0, -2, 1)
  \end{equation*}

  Por lo tanto
  \begin{equation*}
    \iint_{S} (\nabla \times F) \cdot \hat{n} dS = \iint_{D} -4 + 4y dA
  \end{equation*}

  Intersectando las 2 superficies obtenemos que
  \begin{equation*}
    4x^{2} + 9y^{2} \leq 36
  \end{equation*}

  Con el siguiente cambio de coordenadas se tiene que
  \begin{gather*}
    x(r, \theta) = 3 r \cos(\theta)\\
    y(r, \theta) = 2 r \sin(\theta)
  \end{gather*}
  con $r \in [0, 1], \theta \in [0, 2\pi]$

  y el jacobiano es
  \begin{equation*}
    J = \begin{vmatrix}
      3 \cos(\theta) & -3r\sin(\theta)\\
      2 \sin(\theta) & 2r \cos(\theta)
    \end{vmatrix} = 6r
  \end{equation*}

  Por el teorema de cambio de coordenadas
  \begin{equation*}
    \iint_{D} -4 + 4y dA = \int_{0}^{2\pi}\int_{0}^{1} (-4 + 8r \sin\theta)6r dr d\theta = -24\pi
  \end{equation*}

\end{solution}

\begin{problem}
  Considere $C$ la curva de interseccion entre las superficies $S_{1}: x + y + z = 1$ y $S_{2} : z = 2 - x^{2} - y^{2}$. Calcule el trabajo efectuado por el campo de fuerzas
  \begin{equation*}
    F(x, y, z) = (yz, e^{y^{3}}, \cos(z) + y)
  \end{equation*}
  a lo largo de la curva $C$.
\end{problem}
\begin{solution}
  Como ocuparemos el Teorema de Stokes, calcularemos el rotor primero
  \begin{equation*}
    \nabla \times F = \curl{yz}{e^{y^{3}}}{\cos(z) + y} = (1, y, -z)
  \end{equation*}

  Calculemos la curva interseccion
  \begin{gather*}
    1 - x - y = 2 - x^{2} - y^{2}\\
    x^{2} - x + 1 + y^{2} - y = 2\\
    (x - \frac{1}{2})^{2} + (y - \frac{1}{2})^{2} = \frac{3}{2}
  \end{gather*}

  Luego la curva de interseccion esta parametrizada por
  \begin{align*}
    x(t) &= \sqrt{\frac{3}{2}} \cos(t) + \frac{1}{2}\\
    y(t) &= \sqrt{\frac{3}{2}} \sin(t) + \frac{1}{2}\\
    z(t) &= - \sqrt{\frac{3}{2}}(\cos(t) + \sin(t))
  \end{align*}
  con $t \in [0, 2\pi]$. Por el teorema de Stokes
\end{solution}

\begin{problem}
  Determine el trabajo ejercido por el campo vectorial
  \begin{equation*}
    F(x, y, z) = (\cos(x^{2}) - 2y, e^{y} - 2z, \sin(z^{6}) - 2x)
  \end{equation*}
  a lo largo de la curva $C$ que se obtiene de la interseccion del elipsoide $9x^{2} + 3y^{2} + \frac{z^{2}}{4} = 36$ con el plano $z = 2y$
\end{problem}
\begin{solution}
Ocuparemos el teorema de Stokes

Primero calcularemos el rotacional
\begin{equation*}
  \nabla \times F = \curl{\cos(x^{2}) - 2y}{e^{y} - 2z}{\sin(z^{6}) - 2x} = (2, 2, 2)
\end{equation*}

Ocuparemos la parametrizacion natural, luego
\begin{equation*}
  \Phi(x, y) = (x, y, 2y)
\end{equation*}

Entonces tenemos que
\begin{equation*}
  \int_{C} F dr = \iint_{S} \nabla \times F dS
\end{equation*}

Tenemos que la integral de superficie es
\begin{equation*}
  \iint_{S} \nabla \times F dS = \iint_{R_{xy}}(2, 2, 2) \cdot (0, -2, 1) dA = \iint_{R_{xy}} -2 dA = -12\pi
\end{equation*}

\end{solution}

\begin{problem}
  Dado $F(x, y, z) = (\cosh y, zx^{2}, x)$ y $S$ la superficie limitada por la curva $\Gamma$, obtenida de la intersecion
  \begin{equation*}
    S_{1}: x + y = 2 \land S_{2}: x^{2} + y^{2} + z^{2} = 2(x + y)
  \end{equation*}
  orientada contrareloj vista desde el origen. Calcule $\iint_{S} \nabla \times F dS$
\end{problem}
\begin{solution}
  Veamos quien es $\Gamma$
  \begin{align*}
    x^{2} + (2 - x)^{2} + z^{2} &= 4\\
    x^{2} + 4 - 4x + x^{2} + z^{2} &= 4\\
    2x^{2} - 4x + z^{2} &= 0\\
    2(x^{2} - 2x) + z^{2} &= 0\\
    2(x - 1)^{2} + z^{2} &= 2\\
    (x - 1)^{2} + \frac{z^{2}}{2} &= 1
  \end{align*}

  Luego la parametrizacion de esta curva es
  \begin{equation*}
    \Phi(r, \theta) = (r \cos(\theta) + 1, 1 - r \cos(\theta), \sqrt{2}r \sin(\theta)
  \end{equation*}

  Calculemos le rotor
  \begin{equation*}
    \nabla \times F = \curl{\cosh y}{zx^{2}}{x} = (-x^{2}, -1, 2zx - \sinh(y))
  \end{equation*}

  Ahora calculemos la normal
  \begin{equation*}
    \hat{n} = \Phi_{r} \times \Phi_{\theta} = \crossprod{\cos(\theta)}{-\cos(\theta)}{\sqrt{2}\sin(\theta)}{-r \sin(\theta)}{r \sin(\theta)}{\sqrt{2}r \cos(\theta)} = (-\sqrt{2} r, -\sqrt{2} r, 0)
  \end{equation*}

  Entonces
  \begin{gather*}
    \int_{0}^{2\pi}\int_{0}^{1} (r \cos(\theta) + 1)^{2}\sqrt{2}r + \sqrt{2}r dr d\theta =\\
    \int_{0}^{2\pi}\int_{0}^{1} \sqrt{2}r (r^{2} \cos^{2}(\theta) + 2r \cos(\theta) + 2) dr d \theta =\\
    \frac{\pi \sqrt{2}}{4} + 2\sqrt{2} \pi = \frac{9}{4} \sqrt{2} \pi
  \end{gather*}

\end{solution}

\subsection{Teorema de la divergencia}
\begin{problem}
  Usando el teorema de la divergencia calcule $\iint_{S} F \cdot \hat{n} dS$ donde $S$ es la superficie lateral
  del tronco del cono $z = \sqrt{x^{2} + y^{2}}$ limitado por los planos $z = 1$ y $z = 4$ y
  $F(x, y, z) = (x^{2} + 2z, y^{2} + z^{2}, 1)$ y $\hat{n}$ es la normal exterior.
\end{problem}
\begin{solution}
  Consideremos la siguiente superficie $S^{\star} = S \cup S^{T_{1}} \cup S^{T_{2}}$ donde tenemos que
  \begin{gather*}
    S^{T_{1}}: x^{2} + y^{2} \leq 16, z = 4\\
    S^{T_{2}}: x^{2} + y^{2} \leq 1, z = 1
  \end{gather*}

  Dado que $S^{\star}$ es una superficie cerrada, podemos ocupar el teorema de Gauss el cual dice
  \begin{equation*}
    \iint_{S^{\star}} F \cdot \hat{n} dS = \iiint_{V} \nabla \cdot F dV
  \end{equation*}

  Calculemos la divergencia
  \begin{equation*}
    \nabla \cdot F = 2x + 2y
  \end{equation*}

  Calculemos la integral. Calculemos las variaciones en las coordenadas cilindricas
  \begin{gather*}
    0 \leq r \leq z\\
    1 \leq z \leq 4\\
    0 \leq \theta \leq 2\pi
  \end{gather*}
  Y sabemos que el jacobiano de las cilindricas es $r$. Calculemos la integral
  \begin{equation*}
    \iiint_{V} \nabla \cdot F dV = \int_{0}^{2\pi}\int_{1}^{4}\int_{0}^{z} (2r \cos \theta + 2r \sin \theta) r dr dz d \theta = 0
  \end{equation*}

  Por el teorema de Gauss entonces tenemos que
  \begin{equation*}
    \iint_{S} F \cdot \hat{n} dS + \iint_{S^{T_{1}}} F \cdot \hat{n} dS + \iint_{S^{T_{2}}} F \cdot \hat{n} dS = 0
  \end{equation*}

  Calculemos la segunda integral.
  \begin{equation*}
    \iint_{S^{T_{1}}} F \cdot \hat{n} dS = \int_{0}^{2\pi}\int_{0}^{1} -r dr d\theta = -\pi
  \end{equation*}

  Calculemos la tercera integral.
  \begin{equation*}
    \iint_{S^{T_{2}}} F \cdot \hat{n} dS = \int_{0}^{2\pi}\int_{0}^{4} r dr d \theta = 16\pi
  \end{equation*}

  Concluyendo asi que
  \begin{equation*}
    \iint_{S} F \cdot \hat{n} dS = -15\pi
  \end{equation*}
\end{solution}

\begin{problem}
  Sea $F(x, y, z) = (y^{2} - z^{2}, x^{2} - y^{3}, 3zy^{2} + z^{2}e^{x^{2} + y^{2}})$ y $S$ el contorno de la region encerrada por las superficies $x^{2} + y^{2} - z^{2} = 1$, $z = 0$, $z = 3$, calcule $\iint_{s} F dS$
\end{problem}
\begin{solution}
  Cerremos la superficie para poder ocupar el teorema de la divergencia. Definamos la siguiente superficie
  \begin{equation*}
    S^{\star} = S \cup S^{T_{1}} \cup S^{T_{2}}
  \end{equation*}
  donde $S^{T_{1}}: x^{2} + y^{2} \leq 1,  z = 0$ y $S^{T_{2}}: x^{2} + y^{2} \leq 10, z = 3$

  Ahora por la formula de Ostrogradski tenemos que
  \begin{equation*}
    \iint_{S^{\star}} F dS = \iiint_{V} \nabla \cdot F dV
  \end{equation*}

  Calculemos la divergencia
  \begin{equation*}
    \nabla \cdot F = 2ze^{x^{2} + y^{2}}
  \end{equation*}

  Ocupando coordenadas cilindricas
  \begin{gather*}
    x = r \cos (\theta)\\
    y = r \sin (\theta)\\
    z = z
  \end{gather*}

  con $0, \leq z \leq 3$, $0 \leq \theta \leq 2\pi$, $0 \leq r \leq \sqrt{1 + z^{2}}$ y el Jacobiano es $r$ entonces
  \begin{align*}
    \iiint_{V} 2ze^{x^{2} + y^{2}} dV &= \int_{0}^{2\pi}\int_{0}^{3}\int_{0}^{\sqrt{1 + z^{2}}} 2zre^{r^{2}} dr dz d\theta\\
                                      &= 2\pi \int_{0}^{3} z(e^{1 + z^{2}} - 1) dz\\
    &= \fem
  \end{align*}

  \begin{equation*}
    \iint_S F dS + \iint_{S^{T_1}} F dS + \iint_{S^{T_2}} F dS = \fem
  \end{equation*}
\end{solution}

\begin{problem}
Si $\Omega = \{(x, y, z) \in \R^{3} : x^{2} + y^{2} + z^{2} \leq a^{2}, z \geq \sqrt{x^{2} + y^{2}}\}$, donde $a > 0$. Calcule el flujo a traves de la superficie frontera de $\Omega$ en sentido normal exterior a esta del campo, donde el campo es $F(x, y, z) = (x \cos^{2}(z), y \sin^{2}(z), e^{x}\sin(y - x) + z)$
\end{problem}
\begin{solution}
Notemos que la frontera del volumen dado es cerrada, simple regular y suave. $F \in \mathcal{C}^{\infty}$, calculemos la divergencia.
\begin{equation*}
\nabla \cdot F(x, y, z) = \cos^{2}(z) + \sin^{2}(z) + 1 = 2
\end{equation*}

Entonces por el teorema de Gauss tenemos que el flujo es el siguiente
\begin{equation*}
\Phi = \iint_{S} F dS = \iiint_{V} \nabla \cdot F dV = 2 \iiint_{V} dV
\end{equation*}

Aplicando coordenadas esfericas obtenemos
\begin{gather*}
  x = \rho \cos(\theta) \sin(\phi)\\
  y = \rho \sin(\theta) \sin(\phi)\\
  z = \rho \cos(\phi)\\
  J = \rho^{2} \sin(\phi)
\end{gather*}

Entonces $\Omega$ queda de la siguiente forma
\begin{equation*}
  \rho^{2} \leq a^{2} \land \cos(\phi) \geq \sin(\phi)
\end{equation*}

Luego nuestras variaciones son
\begin{gather*}
  0 \leq \theta \leq 2\pi\\
  0 \leq \rho \leq a\\
  0 \leq \phi \leq \frac{\pi}{4}
\end{gather*}

Entonces nuestra solucion es
\begin{equation*}
2 \int_{0}^{2\pi}\int_{0}^{a}\int_{0}^{\frac{\pi}{4}} \rho^{2} \sin(\phi) dV = 4\pi \frac{a^{3}}{3} (1 - \frac{1}{\sqrt{2}})
\end{equation*}
\end{solution}

\begin{problem}
  Considere la superficie $S = S_{1} \cup S_{2}$ donde
  \begin{gather*}
    S_{1} : x^{2} + y^{2} + 2(x - 2y) + 4 \leq 0, z = x + 2\\
    S_{2} : x^{2} + y^{2} + 2(x - 2y) + 4 = 0, x + 2 \leq z \leq 4 + 2x
  \end{gather*}

  Calcule el flujo de $F(x, y, z) = (z, y, x)$ a traves de la superficie $S$.
\end{problem}
\begin{solution}
  Escribamos de otra forma la primera superficie
  \begin{equation*}
    x^{2} + y^{2} + 2x - 4y + 4 \leq 0 \implies x^{2} + y^{2} + 2z - 4y \leq 0 \implies x^{2} + (y - 2)^{2} + 2z \leq 4
  \end{equation*}
  La cual queda de la siguiente forma
  \begin{equation*}
    -\frac{1}{2}(x^{2} + (y - 2)^{2}) + 2 \leq z
  \end{equation*}

  \begin{equation*}
    2x + 4 = -x^{2} -y^{2} + 4y
  \end{equation*}

  Por lo tanto
  \begin{equation*}
    -\frac{1}{2}(x^{2} + y^{2} - 4y) \leq z \leq -(x^{2} + y^{2} - 4y)
  \end{equation*}
\end{solution}

\subsection{Sturm-Liouville}
\begin{problem}
  Resuelva el siguiente problema de Sturm-Liouville
  \begin{equation*}
    \begin{dcases}
      x''(x) - 2x'(x) + \lambda x(x) = 0\\
      x(0) = 0\\
      x'(1) = x(1)
    \end{dcases}
  \end{equation*}
\end{problem}
\begin{solution}
    La ecuacion caracteristica asociada al problema es
    \begin{equation*}
      m^{2} - 2m + \lambda = 0
    \end{equation*}

    Luego las soluciones vienen dadas por
    \begin{equation*}
      m_{1,2} = \frac{2 \pm \sqrt{4 - 4\lambda}}{2} = 1 \pm \sqrt{1 - \lambda}
    \end{equation*}

    \begin{enumerate}
      \item Caso $\lambda < 1$. Tenemos que $1 - \lambda > 0$ por lo tanto la solucion a la EDO
            viene dada por
            \begin{equation*}
              x(x) = Ae^{(1 + \sqrt{1 - \lambda})x} + B e^{(1 - \sqrt{1 - \lambda})x}
            \end{equation*}

            Derivamos
            \begin{equation*}
              x'(x) = A(1 + \sqrt{1 - \lambda})e^{(1 + \sqrt{1 - \lambda})x} + B(1 - \sqrt{1 - \lambda})e^{(1 - \sqrt{1 - \lambda})x}
            \end{equation*}

            Aplicando condiciones iniciales obtenemos
            \begin{gather*}
              A + B = 0\\
              A(1 + \sqrt{1 - \lambda})e^{1 + \sqrt{1 - \lambda}} + B(1 - \sqrt{1 - \lambda})e^{1 - \sqrt{1 - \lambda}} = Ae^{1 + \sqrt{1 - \lambda}} + Be^{1 - \sqrt{1 - \lambda}}
            \end{gather*}

            Moviendo las cosas
            \begin{gather*}
              A\sqrt{1 - \lambda}e^{1 + \sqrt{1 - \lambda}}  - B\sqrt{1 - \lambda}e^{1 - \sqrt{1 - \lambda}} = 0\\
              A(\sqrt{1 - \lambda}e^{1 + \sqrt{1 - \lambda}} + \sqrt{1 - \lambda}e^{1 - \sqrt{1 - \lambda}}) = 0\\
              A = 0 \implies B = 0
            \end{gather*}
      \item Caso $\lambda > 1$. Tenemos que $1 - \lambda < 0$ por lo tanto la solucion a la EDO viene dada por
            \begin{equation*}
              x(x) = e^{x}(A \cos(\sqrt{\lambda - 1} x) + B \sin(\sqrt{\lambda - 1}x))
            \end{equation*}

            Aplicando la primera condicion inicial
            \begin{equation*}
              A = 0
            \end{equation*}

            entonces
            \begin{equation*}
              x(x) = B e^{x} \sin(\sqrt{\lambda - 1}x)
            \end{equation*}

            luego
            \begin{equation*}
              x'(x) = Be^{x} \sin(\sqrt{\lambda - 1} x) + Be^{x} \sqrt{\lambda - 1}\cos(\sqrt{\lambda - 1} x)
            \end{equation*}

            ocupando la segunda condicion inicial
            \begin{equation*}
              Be^{x}\sqrt{\lambda - 1} \cos(\sqrt{\lambda - 1}) = 0
            \end{equation*}

            Como estamos buscando soluciones no nulas
            \begin{equation*}
              \cos(\sqrt{\lambda - 1}) = 0 \implies \sqrt{\lambda - 1} = \frac{\pi}{2} + k \pi
            \end{equation*}

            Por lo tanto los valores propios son
            \begin{equation*}
              \lambda_{n} = (\frac{(1 + 2n)\pi}{2})^{2} + 1
            \end{equation*}
            y las funciones propias son
            \begin{equation*}
              x_{n}(x) = A_{n}\sin(\sqrt{\lambda_{n} - 1}x)
            \end{equation*}

      \item Caso $\lambda = 1$. Luego la solucion a la edo es
            \begin{equation*}
              x(x) = Ae^{x} + Bxe^{x}
            \end{equation*}

            Aplicando la primera condicion inicial
            \begin{equation*}
              A = 0
            \end{equation*}

            Por lo tanto
            \begin{equation*}
              x(x) = Bxe^{x} \implies x'(x) = Be^{x} + Bxe^{x}
            \end{equation*}

            Con la segunda condicion inicial tenemos
            \begin{equation*}
              B = 0
            \end{equation*}

            Soluciones triviales.
    \end{enumerate}

    Por lo tanto las soluciones son
    \begin{equation*}
      x_{n}(x) = A_{n}\sin(\sqrt{\lambda_{n} - 1}x)
    \end{equation*}
    \begin{equation*}
      \lambda_{n} = (\frac{(1 + 2n)\pi}{2})^{2} + 1
    \end{equation*}

\end{solution}

\subsection{EDP}
\begin{problem}
  Resuelva la siguiente EDP mediante la tecnica de separacion de variables
  \begin{equation*}
    \begin{dcases}
      v_{t} &= v_{xx}\\
      v(0, t) &= 0\\
      v_{x}(2, t) &= 0\\
      v(x, 0) &= 5 \sin(\frac{3\pi x}{4})
    \end{dcases}
  \end{equation*}
\end{problem}
\begin{solution}
  Por el metodo de separacion de variables planteamos la siguiente solucion.
  \begin{equation*}
    v(x, t) = X(x)T(t)
  \end{equation*}
  Reemplazamos en la primera ecuacion
  \begin{equation*}
    XT' = X''T \implies \frac{T'}{T} = \frac{X''}{X} = - \lambda
  \end{equation*}

  Entonces tenemos la siguiente EDO.
  \begin{equation*}
    \frac{T'}{T} = -\lambda \implies T_{n}(t) = A_{n}e^{-\lambda_{n} t}
  \end{equation*}

  Y obtenemos el siguiente problema de Sturm-Liouville
  \begin{equation*}
    \begin{dcases}
      X'' + \lambda X &= 0\\
      X(0) &= 0\\
      X'(2) &= 0
    \end{dcases}
  \end{equation*}

  Donde la solucion vienen dada por
  \begin{gather*}
    \lambda_{n} = (\frac{(n - \frac12)\pi}{2})^{2}\\
    X_{n}(x) = \sin(\frac{(n - \frac12)\pi}{2} x)
  \end{gather*}

  Entonces la solucion formal a nuestra EDP es
  \begin{equation*}
    v(x, t) = \sum_{n = 1}^{\infty} A_{n}e^{-(\frac{(n - \frac12)\pi}{2})^{2} t} \sin(\frac{(n - \frac12)\pi}{2} x)
  \end{equation*}

  Aplicando la condicion inicial obtenemos que
  \begin{equation*}
    5 \sin (\frac{3\pi}{4}x) = \sum_{n = 1}^{\infty} A_{n} \sin(\frac{(n - \frac12)\pi}{2} x)
  \end{equation*}
  Por la ortogonalidad de las eigenfunciones obtenemos que todos los $A_{n} = 0$ excepto cuando
  \begin{equation*}
    \frac{3 \pi}{4} = \frac{(n - \frac{1}{2})\pi}{2} \implies n = 2
  \end{equation*}

  en cuyo caso tenemos que $A_{2} = 5$. Por lo tanto la solucion a nuestra EDP es
  \begin{equation*}
    v(x, t) = 5 e^{-\frac{9\pi^{2}}{16} t} \sin(\frac{3\pi}{4} x)
  \end{equation*}
\end{solution}

\begin{problem}
  Resuelva la siguiente EDP mediante la tecnica de separacion de variables
  \begin{equation*}
  \begin{dcases}
    u_{tt} &= u_{xx} - u_t\\
    u_x(0, t) &= 0\\
    u(\pi, t) &= 0\\
    u(x, 0) &= 0\\
    u_t(x, 0) &= 3 \cos(\frac{5\pi}{2})
  \end{dcases}
  \end{equation*}
\end{problem}
\begin{solution}
  Por el metodo de separacion de variables tenemos que
  \begin{equation*}
    u(x, t) = X(x)T(t)
  \end{equation*}

  Reemplazando en la primera ecuacion obtenemos
  \begin{equation*}
    XT'' = X''T - XT' \implies XT'' + XT' = X''T \implies \frac{T''}{T} + \frac{T'}{T} = \frac{X''}{X} = -\lambda
  \end{equation*}

  Resolvamos primero la EDO que nos queda en $T$
  \begin{equation*}
    T'' + T' + \lambda T = 0
  \end{equation*}

  esto es una EDO lineal de segundo orden, resolveremos mediante el polinomio caracteristico.
  \begin{equation*}
    m^{2} + m + \lambda = 0
  \end{equation*}

  donde las soluciones son
  \begin{equation*}
    m = \frac{-1 \pm \sqrt{1 - 4\lambda}}{2}
  \end{equation*}

  Por lo tanto necesitamos saber el valor de $\lambda$. Resolvamos el problema de Sturm-Liouville
  \begin{equation*}
    \begin{dcases}
      X'' + \lambda X &= 0\\
      X'(0) &= 0\\
      X(\pi) &= 0
    \end{dcases}
  \end{equation*}
  Donde la solucion viene dada por
  \begin{gather*}
    \lambda_{n} = (n - \frac12)^{2}\\
    X_{n}(x) = \cos((n - \frac12)x)
  \end{gather*}
  Volviendo a la EDO anterior obtenemos que dado que $\lambda_{n} \geq \frac{1}{4}$ las soluciones son
  \begin{gather*}
    T_{1}(t) = Ae^{-\frac12 t} + Bxe^{-\frac12 t}\\
    T_{n}(t) = e^{-\frac12 t}(A \cos(\sqrt{n^{2} - n} t) + B \sin(\sqrt{n^{2} - n}t))
  \end{gather*}

  Luego la solucion formal a nuestra EDP es
\end{solution}
\end{document}
