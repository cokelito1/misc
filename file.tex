%% 
%% This is file, `file.tex',
%% generated with the extract package.
%% 
%% Generated on :  2023/07/05,20:55
%% From source  :  main.tex
%% Using options:  active,generate=file,extract-env={problem}
%% 
\documentclass{article}

\begin{document}

\begin{problem}
Usando el teorema de Stokes, calcular la integral de linea $\oint_{C} x^{2}y^{3} dx + dy + z dz$ donde $C$ es la curva $x^{2} + y^{2} = R^{2}, z = 0$ con $R > 0$, recorrida en sentido antihorario
\end{problem}

\begin{problem}
Calcule $\oint_{C} x \sin x - 2y^{2} dx + y \cos y - 2z dy + \tan z - 2x dz$ donde $C$ es la interseccion de $4x^{2} + 5y^{2} + z^{2} = 36$ con $z = 2y$
\end{problem}

\begin{problem}
  Considere $C$ la curva de interseccion entre las superficies $S_{1}: x + y + z = 1$ y $S_{2} : z = 2 - x^{2} - y^{2}$. Calcule el trabajo efectuado por el campo de fuerzas
  \begin{equation*}
    F(x, y, z) = (yz, e^{y^{3}}, \cos(z) + y)
  \end{equation*}
  a lo largo de la curva $C$.
\end{problem}

\begin{problem}
  Determine el trabajo ejercido por el campo vectorial
  \begin{equation*}
    F(x, y, z) = (\cos(x^{2}) - 2y, e^{y} - 2z, \sin(z^{6}) - 2x)
  \end{equation*}
  a lo largo de la curva $C$ que se obtiene de la interseccion del elipsoide $9x^{2} + 3y^{2} + \frac{z^{2}}{4} = 36$ con el plano $z = 2y$
\end{problem}

\begin{problem}
  Dado $F(x, y, z) = (\cosh y, zx^{2}, x)$ y $S$ la superficie limitada por la curva $\Gamma$, obtenida de la intersecion
  \begin{equation*}
    S_{1}: x + y = 2 \land S_{2}: x^{2} + y^{2} + z^{2} = 2(x + y)
  \end{equation*}
  orientada contrareloj vista desde el origen. Calcule $\iint_{S} \nabla \times F dS$
\end{problem}

\begin{problem}[Certamen MAT024 2016-2]
  Determine la magnitud de la circulacion del campo
  \begin{equation*}
    F(x, y, z) = (x \cos(x^{2}) - y, y \sin(y^{3}) - z, h(z) - x), h \in \mathcal{C^{1}}
  \end{equation*}
  a lo largo de la curva $C$ que se obtiene de la interseccion del elipsoide $\frac{x^{2}}{16} + \frac{y^{2}}{9} + \frac{z^{2}}{4} = 1$ con el plano $y = 2z - x + 1$
\end{problem}

\begin{problem}[Precertamen MAT024 2019]
  Usar el teorema de Stokes para evaluar la integral de linea
  \begin{equation*}
    \int_{C} -y^{3} dx + x^{3} dy - z^{3} dz
  \end{equation*}
  donde $C$ es la interseccion del cilindro $x^{2} + y^{2} = 1$ y el plano $x + y + z = 1$, y la
  orientacoin de $C$ es en sentido contrario a las manecillas del reloj, en el plano $xy$.
\end{problem}

\begin{problem}[Precertamen MAT024 2019]
  Sea $C$ una curva cerrada simple la cual es borde de una superficie $S$ de area $\lambda$, orientada respecto a la normal $\hat{n} = (1, 0, 1)$. Calcule el trabajo de un campo vectorial
  $F$ cuyo rotacional es $\nabla \times F = (1, 1, 1)$ a lo largo de $C$.
\end{problem}

\begin{problem}[Precertamen MAT024 2019]
  Considere la curva $C$ definida por la interseccion de las superficies
  \begin{equation*}
    S_{1}: 9x^{2} + 4y^{2} = 36, S_{2} : z = 2x
  \end{equation*}
  y el campo $F(x, y, z) = (z \cos(y^{2}) - 2z, -2xyz \sin(y^{2}) - 2x, x \cos(y^{2}) -2y)$, encuentre el trabajo realizado por $F$ a lo largo de la curva.
\end{problem}

\begin{problem}
  Usando el teorema de la divergencia calcule $\iint_{S} F \cdot \hat{n} dS$ donde $S$ es la superficie lateral
  del tronco del cono $z = \sqrt{x^{2} + y^{2}}$ limitado por los planos $z = 1$ y $z = 4$ y
  $F(x, y, z) = (x^{2} + 2z, y^{2} + z^{2}, 1)$ y $\hat{n}$ es la normal exterior.
\end{problem}

\begin{problem}
  Sea $F(x, y, z) = (y^{2} - z^{2}, x^{2} - y^{3}, 3zy^{2} + z^{2}e^{x^{2} + y^{2}})$ y $S$ el contorno de la region encerrada por las superficies $x^{2} + y^{2} - z^{2} = 1$, $z = 0$, $z = 3$, calcule $\iint_{s} F dS$
\end{problem}

\begin{problem}
Si $\Omega = \{(x, y, z) \in \R^{3} : x^{2} + y^{2} + z^{2} \leq a^{2}, z \geq \sqrt{x^{2} + y^{2}}\}$, donde $a > 0$. Calcule el flujo a traves de la superficie frontera de $\Omega$ en sentido normal exterior a esta del campo, donde el campo es $F(x, y, z) = (x \cos^{2}(z), y \sin^{2}(z), e^{x}\sin(y - x) + z)$
\end{problem}

\begin{problem}
  Considere la superficie $S = S_{1} \cup S_{2}$ donde
  \begin{gather*}
    S_{1} : x^{2} + y^{2} + 2(x - 2y) + 4 \leq 0, z = x + 2\\
    S_{2} : x^{2} + y^{2} + 2(x - 2y) + 4 = 0, x + 2 \leq z \leq 4 + 2x
  \end{gather*}

  Calcule el flujo de $F(x, y, z) = (z, y, x)$ a traves de la superficie $S$.
\end{problem}

\begin{problem}
  Calcular $\iint_{S} F dS$ donde $S$ es la superficie $x^{2} + y^{2} + z^{2} - 2\sqrt{x^{2} + y^{2}} = 0$ con $z \geq 0$ y $F(x, y, z) = (x, y, z)$
\end{problem}

\begin{problem}[Precertamen MAT024 2019]
  Calcule el flujo del campo vectorial \begin{equation*} F(x, y, z) = (2x, z - \frac{zx}{x^{2} + y^{2} + z^{2}}, \frac{xy}{x^{2} + y^{2} + z^{2}})\end{equation*} a traves de la superficie $S$ descrita por
  \begin{equation*}
    S: \frac{(x - 1)^{2}}{4} + \frac{(y - 1)^{2}}{9} + (z - 2)^{2} = 1
  \end{equation*}
  orientada respecto a la normal unitaria exterior.
\end{problem}

\begin{problem}[Precertamen MAT024 2019]
  Considere el campo vectorial
  \begin{equation*}
    F(x, y, z) = (x, x(y - 1), xyz)
  \end{equation*}
  y la superficie $S$ definida como
  \begin{equation*}
    S: x^{2} + y^{2}  = 16, 0 \leq z \leq 4 - y
  \end{equation*}
  Calcule el flujo a traves de $S$ con respeto a la normal exterior.
\end{problem}

\begin{problem}[Precertamen MAT024 2019]
  Considere las superficie $S_{1}$ y $S_{2}$ definidas como
  \begin{gather*}
    S_{1}: x^{2} + y^{2} \leq 4, z = 1\\
    S_{2}: x^{2} + y^{2} = 4, 1 \leq z \leq 5.
  \end{gather*}

  Determinar el flujo del campo vectorial
  \begin{equation*}
    F(x, y, z) = (y^{2}, x^{2}, z)
  \end{equation*}
  a traves de $S = S_{1} \cup S_{2}$
\end{problem}

\begin{problem}
  Resuelva el siguiente problema de Sturm-Liouville
  \begin{equation*}
    \begin{dcases}
      x''(x) - 2x'(x) + \lambda x(x) = 0\\
      x(0) = 0\\
      x'(1) = x(1)
    \end{dcases}
  \end{equation*}
\end{problem}

\begin{problem}
  Resuelva la siguiente EDP mediante la tecnica de separacion de variables
  \begin{equation*}
    \begin{dcases}
      v_{t} &= v_{xx}\\
      v(0, t) &= 0\\
      v_{x}(2, t) &= 0\\
      v(x, 0) &= 5 \sin(\frac{3\pi x}{4})
    \end{dcases}
  \end{equation*}
\end{problem}

\begin{problem}
  Resuelva la siguiente EDP mediante la tecnica de separacion de variables
  \begin{equation*}
  \begin{dcases}
    u_{tt} &= u_{xx} - u_t\\
    u_x(0, t) &= 0\\
    u(\pi, t) &= 0\\
    u(x, 0) &= 0\\
    u_t(x, 0) &= 3 \cos(\frac{5\pi}{2})
  \end{dcases}
  \end{equation*}
\end{problem}

\begin{problem}[Precertamen 2020 MAT024]
  Resuelva mediante el metodo de separacion de variables la siguiente EDP.
  \begin{equation*}
  \begin{dcases}
    u_t = u_{xx} - u, & 0 \leq x \leq \pi, t \geq 0\\
    u_x(0, t) = u(\pi, t) = 0, & t > 0\\
    u(x, 0) = \sin(x) & 0 < x < \pi
  \end{dcases}
  \end{equation*}
\end{problem}

\begin{problem}[Certamen 3 2020 MAT024]
  Resuelva mediante el metodo de separacion de variables
  \begin{equation*}
  \begin{dcases}
    u_t &= u_{xx} - 6x\\
    u(0, t) &= 3\\
    u_{x}(2, t) &= 2\\
    u(x, 0) &= x^{3} - 10x + 3 + 5 \sin(\frac{3\pi x}{4})
  \end{dcases}
  \end{equation*}
\end{problem}

\begin{problem}
  Sea $(X, \topo)$ un espacio topologico tal que $B \subset X$ sea un subconjunto denso en $X$. Si
  $A$ es un conjunto denso en $(B, \topo_{B})$, donde $\topo_{B}$ es la topologia inducida de $X$ en $B$, demostrar que $A$ es denso en $(X, \topo)$
\end{problem}

\end{document}
