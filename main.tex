\documentclass{article}

\usepackage{personal}
\usepackage{amsmath}
\usepackage{amsfonts}
\usepackage{amsthm}
\usepackage{chngcntr}
\usepackage{mathtools}

\title{Pura mierda}
\author{Jorge Eduardo Bravo Soto}
\date{\today}

\theoremstyle{definition}
\newtheorem{problem}{Problema}
\counterwithin*{problem}{subsection}

\newenvironment{solution}{\begin{proof}[Solucion]}{\end{proof}}

\begin{document}
\maketitle
\tableofcontents

\section{MAT024}
\subsection{Teorema de Stokes}
\begin{problem}
Usando el teorema de Stokes, calcular la integral de linea $\oint_{C} x^{2}y^{3} dx + dy + z dz$ donde $C$ es la curva $x^{2} + y^{2} = R^{2}, z = 0$ con $R > 0$, recorrida en sentido antihorario
\end{problem}
\begin{solution}
  Notemos que la curva $C$ es cerrada, simple y suave. Esta curva encierra a la superficie $S: x^{2} + y^{2} \leq R^{2}, z = 0$ entonces por el teorema de Stokes tenemos que
  \begin{equation*}
    \iint_{S} \nabla \times F dS = \oint_{C} F dr
  \end{equation*}
  Calculemos $\nabla \times F$
  \begin{equation*}
    \nabla \times F = \curl{x^{2}y^{3}}{1}{z}
  \end{equation*}

  Consideremos la siguiente $\mathcal{C}^{\infty}$ parametrizacion de $S$
  \begin{gather*}
    \phi : [0, 2\pi] \times [0, R] \to \R^{3}\\
    (\theta, r) \mapsto (r \cos \theta, r \sin \theta, 0)
  \end{gather*}

  Donde tenemos que
  \begin{equation*}
    \hat{n} = \phi_{\theta} \times \phi_{r}
  \end{equation*}

  \begin{gather*}
    \phi_{\theta} = (-r \sin \theta, r \cos \theta, 0)\\
    \phi_{r} = (\cos \theta, \sin \theta, 0)
  \end{gather*}

  Por lo tanto el vector normal es
  \begin{equation*}
    \hat{n} = \crossprod{-r \sin \theta}{r \cos \theta}{0}{\cos \theta}{\sin \theta}{0} = (0, 0, -r)
  \end{equation*}

  Por lo tanto tenemos que
  \begin{equation*}
    \oint_{C} F \cdot dr = \iint_{D} (0, 0, 3r^{2}\cos^{2}(\theta) r^{2} \sin^{2}(\theta)) \cdot (0, 0, -r) dA
  \end{equation*}

  Calculemos la integral, donde el dominio es el dominio de la parametrizacion entonces
  \begin{align*}
    \int_{0}^{2\pi}\int_{0}^{R} -3r^{5}\cos^{2}(\theta)\sin^{2}(\theta) drd\theta &= -\frac{R^{6}}{2} \int_{0}^{2\pi}\cos^{2}(\theta)\sin^{2}(\theta) d\theta\\
                                                                                  &= -\frac{R^{6}}{2} \int_{0}^{2\pi} \cos^{2}(\theta)(1 - \cos^{2} \theta) d\theta\\
                                                                                  &= -\frac{R^{6}}{2} (\pi - \int_{0}^{2\pi} \cos^{4} \theta d \theta)\\
    &= -\frac{R^{6}\pi}{8}
  \end{align*}
\end{solution}

\begin{problem}
Calcule $\oint_{C} x \sin x - 2y^{2} dx + y \cos y - 2z dy + \tan z - 2x dz$ donde $C$ es la interseccion de $4x^{2} + 5y^{2} + z^{2} = 36$ con $z = 2y$
\end{problem}
\begin{solution}
  Notemos que $C$ es una curva cerrada, simple y suave. Podemos ocupar el teorema de Stokes por lo
  tanto
  \begin{equation*}
    \oint_{C} F dr = \iint_{S} (\nabla \times F) dS
  \end{equation*}

  Calculemos el rotor
  \begin{equation*}
    \nabla \times F = \curl{x \sin x - 2y^{2}}{y \cos y - 2z}{\tan z - 2x} = (2, 2, 4y)
  \end{equation*}

  Consideremos la siguiente parametrizacion, con las variaciones por determinar
  \begin{equation*}
    \phi(x, y) = (x, y, 2y)
  \end{equation*}

  donde sabemos que la normal es
  \begin{equation*}
    \hat{n} = (-f_{x}, -f_{y}, 1) = (0, -2, 1)
  \end{equation*}

  Por lo tanto
  \begin{equation*}
    \iint_{S} (\nabla \times F) \cdot \hat{n} dS = \iint_{D} -4 + 4y dA
  \end{equation*}

  Intersectando las 2 superficies obtenemos que
  \begin{equation*}
    4x^{2} + 9y^{2} \leq 36
  \end{equation*}

  Con el siguiente cambio de coordenadas se tiene que
  \begin{gather*}
    x(r, \theta) = 3 r \cos(\theta)\\
    y(r, \theta) = 2 r \sin(\theta)
  \end{gather*}
  con $r \in [0, 1], \theta \in [0, 2\pi]$

  y el jacobiano es
  \begin{equation*}
    J = \begin{vmatrix}
      3 \cos(\theta) & -3r\sin(\theta)\\
      2 \sin(\theta) & 2r \cos(\theta)
    \end{vmatrix} = 6r
  \end{equation*}

  Por el teorema de cambio de coordenadas
  \begin{equation*}
    \iint_{D} -4 + 4y dA = \int_{0}^{2\pi}\int_{0}^{1} (-4 + 8r \sin\theta)6r dr d\theta = -24\pi
  \end{equation*}

\end{solution}

\subsection{Teorema de la divergencia}
\begin{problem}
  Usando el teorema de la divergencia calcule $\iint_{S} F \cdot \hat{n} dS$ donde $S$ es la superficie lateral
  del tronco del cono $z = \sqrt{x^{2} + y^{2}}$ limitado por los planos $z = 1$ y $z = 4$ y
  $F(x, y, z) = (x^{2} + 2z, y^{2} + z^{2}, 1)$ y $\hat{n}$ es la normal exterior.
\end{problem}
\begin{solution}
  Consideremos la siguiente superficie $S^{\star} = S \cup S^{T_{1}} \cup S^{T_{2}}$ donde tenemos que
  \begin{gather*}
    S^{T_{1}}: x^{2} + y^{2} \leq 16, z = 4\\
    S^{T_{2}}: x^{2} + y^{2} \leq 1, z = 1
  \end{gather*}

  Dado que $S^{\star}$ es una superficie cerrada, podemos ocupar el teorema de Gauss el cual dice
  \begin{equation*}
    \iint_{S^{\star}} F \cdot \hat{n} dS = \iiint_{V} \nabla \cdot F dV
  \end{equation*}

  Calculemos la divergencia
  \begin{equation*}
    \nabla \cdot F = 2x + 2y
  \end{equation*}

  Calculemos la integral. Calculemos las variaciones en las coordenadas cilindricas
  \begin{gather*}
    0 \leq r \leq z\\
    1 \leq z \leq 4\\
    0 \leq \theta \leq 2\pi
  \end{gather*}
  Y sabemos que el jacobiano de las cilindricas es $r$. Calculemos la integral
  \begin{equation*}
    \iiint_{V} \nabla \cdot F dV = \int_{0}^{2\pi}\int_{1}^{4}\int_{0}^{z} (2r \cos \theta + 2r \sin \theta) r dr dz d \theta = 0
  \end{equation*}

  Por el teorema de Gauss entonces tenemos que
  \begin{equation*}
    \iint_{S} F \cdot \hat{n} dS + \iint_{S^{T_{1}}} F \cdot \hat{n} dS + \iint_{S^{T_{2}}} F \cdot \hat{n} dS = 0
  \end{equation*}

  Calculemos la segunda integral.
  \begin{equation*}
    \iint_{S^{T_{1}}} F \cdot \hat{n} dS = \int_{0}^{2\pi}\int_{0}^{1} -r dr d\theta = -\pi
  \end{equation*}

  Calculemos la tercera integral.
  \begin{equation*}
    \iint_{S^{T_{2}}} F \cdot \hat{n} dS = \int_{0}^{2\pi}\int_{0}^{4} r dr d \theta = 16\pi
  \end{equation*}

  Concluyendo asi que
  \begin{equation*}
    \iint_{S} F \cdot \hat{n} dS = -15\pi
  \end{equation*}
\end{solution}

\subsection{EDP}

\section{MAT225}
\subsection{Topologia}

\section{MAT210}
\subsection{Teorema Espectral}

\end{document}
